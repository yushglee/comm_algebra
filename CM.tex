\documentclass[leqno]{amsart}
\usepackage{amssymb}
\usepackage{amsmath} 
\usepackage{enumitem}
\usepackage{hyperref}
\usepackage{mathrsfs}
\usepackage{color}
\usepackage{mathtools,caption,bbm,euscript}
\usepackage[table,dvipsnames]{xcolor}
\usepackage{tikz-cd}
\usepackage[utf8]{inputenc}
\usepackage[OT2,T1]{fontenc}
\hypersetup{
 colorlinks=true,
 linkcolor=DarkOrchid,
 filecolor=blue,
 citecolor=olive,
 urlcolor=orange,
 pdftitle={Complex Multiplication},
 %pdfpagemode=FullScreen,
 }
\usepackage{booktabs}

%[label=(\alph*)]
%[label=(\Alph*)]
%[label=(\roman*)]
%[label={(\bfseries R\arabic*)}]


\setlength{\textwidth}{\paperwidth}
\addtolength{\textwidth}{-2in}
\calclayout


\newcommand{\smat}[1]{\left( \begin{smallmatrix} #1 \end{smallmatrix} \right)}
\newcommand{\mat}[1]{\left( \begin{smallmatrix} #1 \end{smallmatrix} \right)}
\newcommand{\dBr}[1]{\llbracket{#1}\rrbracket}
\newcommand{\leg}[2]{\left(\frac{#1}{#2}\right)}

% double bracket
\makeatletter
\newsavebox{\@brx}
\newcommand{\llangle}[1][]{\savebox{\@brx}{\(\m@th{#1\langle}\)}%
  \mathopen{\copy\@brx\kern-0.5\wd\@brx\usebox{\@brx}}}
\newcommand{\rrangle}[1][]{\savebox{\@brx}{\(\m@th{#1\rangle}\)}%
  \mathclose{\copy\@brx\kern-0.5\wd\@brx\usebox{\@brx}}}
  \newcommand{\llbracket}[1][]{\savebox{\@brx}{\(\m@th{#1[}\)}%
  \mathopen{\copy\@brx\kern-0.5\wd\@brx\usebox{\@brx}}}
\newcommand{\rrbracket}[1][]{\savebox{\@brx}{\(\m@th{#1]}\)}%
  \mathclose{\copy\@brx\kern-0.5\wd\@brx\usebox{\@brx}}}
\makeatother

%%% Unitary group specific
\newcommand{\V}{\mathbf V} 
\newcommand{\W}{\mathbf W} 
\newcommand{\G}{\mathbf G} %GU(2,2)
\newcommand{\X}{\mathbf H} %Hermitian symmetric domain
\newcommand{\KK}{\mathbf K} %compact open subgroup
\newcommand{\xx}{\mathbf x}
\newcommand{\yy}{\mathbf y}
\newcommand{\nn}{\mathbf n} %unipotent

\newcommand{\qdr}[1]{\underline{ #1 }}
\newcommand{\cA}{\mathcal A} %complex abelian varieties
\newcommand{\cB}{\mathcal B} %complex abelian varieties
\newcommand{\bB}{\mathbf B}

\newcommand{\cO}{\mathcal O} %complex abelian varieties
\newcommand{\mm}{\mathbf{m}}


\newcommand{\balpha}{\boldsymbol{\alpha}} %diagonal torus elements
\newcommand{\bbeta}{\boldsymbol{\beta}} %diagonal torus elements
\newcommand{\bnu}{\boldsymbol{\nu}} %character on diagonal torus
\newcommand{\bmu}{\boldsymbol{\mu}} %character on diagonal torus

%%% modular form specific

\newcommand{\wt}[1]{\underline{ #1 }}
\newcommand{\bwt}[1]{\underline{\boldsymbol { #1 }}}
\newcommand{\M}{\mathbf{M}}
\newcommand{\Mk}{\mathbf{M}_{\wt{k}}}
\newcommand{\pM}{\widehat{\mathbf{M}}}
\newcommand{\pMk}{\widehat{\mathbf{M}}_{\wt{k}}}
\newcommand{\AM}{\mathcal{A}}
\newcommand{\bun}{\mathcal{E}}
\newcommand{\ii}{\mathbf i}
\newcommand{\ff}{\mathbf f}
\newcommand{\her}{\EuScript{H} }



\newcommand{\phk}{\mathcal{H}_{\wt{k}}} %Hecke algebra
\newcommand{\ordp}{e}


\newcommand{\vk}{v_{-\wt{k}}} % lowest weight vector
\newcommand{\lk}{\mathnormal{l}_{\wt{k}}} % projection to lowest weight
\newcommand{\blk}{\mathnormal{l}_{\bwt{k}}} % projection to lowest weight
\newcommand{\Lk}{L_{\wt{k}}} 
\newcommand{\pf}{\hat{f}} % algebraic modular form
\newcommand{\pfk}{\hat{f}_{\wt{k}}}
\newcommand{\euF}{\EuScript{F}} % Hida family
\newcommand{\B}{\mathbf{B}} % pairing of algebraic modular form


\newcommand{\ww}{\boldsymbol{\omega}}
\DeclareMathOperator{\an}{an}
\DeclareMathOperator{\ord}{ord}
\DeclareMathOperator{\cts}{cts}
\DeclareMathOperator{\FJ}{FJ}


%%% Linear algebraic groups
\DeclareMathOperator{\GL}{GL}
\DeclareMathOperator{\SL}{SL}
\DeclareMathOperator{\Sp}{Sp}
\DeclareMathOperator{\Mp}{Mp}
\DeclareMathOperator{\GSp}{GSp}
\DeclareMathOperator{\UU}{U}
\DeclareMathOperator{\GUU}{GU}
\DeclareMathOperator{\gl}{\mathfrak{gl}}
\DeclareMathOperator{\mtr}{tr}
\DeclareMathOperator{\diag}{diag}
\DeclareMathOperator{\Ad}{Ad}
\DeclareMathOperator{\vol}{vol}

\DeclareMathOperator{\val}{val}
\DeclareMathOperator{\Lie}{Lie}
\DeclareMathOperator{\Pol}{Pol_p}

%%% Adelic rings
\newcommand{\Q}{{\mathbf{Q}}}
\newcommand{\Z}{{\mathbf{Z}}}
\newcommand{\Qp}{\mathbf{Q}_p}
\newcommand{\Zp}{\mathbf{Z}_p}
\newcommand{\Ql}{\mathbf{Q}_\ell}
\newcommand{\Zl}{\mathbf{Z}_\ell}
\newcommand{\R}{\mathbf R}
\newcommand{\C}{\mathbf C}
\newcommand{\A}{\mathbf A}
\newcommand{\hZ}{{\hat{\mathbf{Z}}}}
\newcommand{\dd}{\mathfrak{d}} %different
\newcommand{\DD}{\mathcal{D}}  %discriminant

\newcommand{\arch}{\mathbf{a}}
\newcommand{\fin}{\mathbf{h}}

\newcommand{\F}{{\mathcal{F}}} %global 
\newcommand{\OF}{{\mathcal{O}_{\F}}}
\newcommand{\K}{{\mathcal{K}}} %global quadratic
\newcommand{\OK}{\mathcal{O}_{\K}}
\newcommand{\kk}{F} %local
\newcommand{\E}{E} %local quadratic


\DeclareMathOperator{\Sel}{Sel}
\DeclareMathOperator{\Gal}{Gal}
\DeclareMathOperator{\Nr}{\mathsf{N}}
\DeclareMathOperator{\Tr}{Tr}
\newcommand{\qch}{\epsilon} % quadratic character of K/F


%%% Fonts
\newcommand{\oeu}{\EuScript{O}}
\newcommand{\eeu}{\EuScript{E}}
\newcommand{\feu}{\EuScript{F}}
\newcommand{\geu}{\EuScript{G}}
\newcommand{\keu}{\EuScript{K}}

\newcommand{\oo}{\mathcal O}
\newcommand{\bs}{\mathcal S}
\newcommand{\id}{\mathbf{1}}

\newcommand{\1}{\mathbf{1}} 
\newcommand{\bfe}{\mathbf e}
\newcommand{\bff}{\mathbf f}

\newcommand{\bX}{\mathbb{X}}
\newcommand{\bY}{\mathbb{Y}}
\newcommand{\bV}{\mathbb{V}}
\newcommand{\bW}{\mathbb{W}}

\newcommand{\fa}{\mathfrak a}
\newcommand{\fg}{\mathfrak g}
\newcommand{\fc}{\mathfrak c}
\newcommand{\fs}{\mathfrak s}
\newcommand{\fm}{\mathfrak m}
\newcommand{\fn}{\mathfrak n}
\newcommand{\fl}{\mathfrak l}
\newcommand{\fp}{\mathfrak p}
\newcommand{\bfp}{\overline{\mathfrak p}}
\newcommand{\fq}{\mathfrak q}
\newcommand{\bfq}{\overline{\mathfrak q}}

\newcommand{\btheta}{\boldsymbol{\theta}}
\newcommand{\bdelta}{\boldsymbol{\delta}}


\newcommand{\fG}{\mathfrak{G}}
\newcommand{\fX}{\mathfrak{X}}
\newcommand{\euW}{\EuScript{W}}


%%% Categorical
\DeclareMathOperator{\Ext}{Ext}
\DeclareMathOperator{\End}{End}
\DeclareMathOperator{\Hom}{Hom}
\DeclareMathOperator{\Inj}{Inj}
\DeclareMathOperator{\Isom}{Isom}
\DeclareMathOperator{\Aut}{Aut}
\DeclareMathOperator{\Ind}{Ind}
\DeclareMathOperator{\coker}{coker}
\DeclareMathOperator{\rank}{rank}
\DeclareMathOperator{\corank}{corank}


\DeclareMathOperator{\Frob}{Frob}
\DeclareMathOperator{\Res}{Res}
\DeclareMathOperator{\rec}{rec}



\newcommand{\bw}{{w^c}}
\newcommand{\ee}{\mathbf e}


\newtheorem*{theorem*}{Theorem}
\newtheorem{thm}{Theorem}[section]
\newtheorem{lem}[thm]{Lemma}
\newtheorem{prop}[thm]{Proposition}
\newtheorem{cor}[thm]{Corollary}


\theoremstyle{definition}
\newtheorem{definition}[thm]{Definition}
\newtheorem{defn}[thm]{Definition}
\theoremstyle{remark}
\newtheorem{rem}[thm]{Remark}
\newtheorem*{Remark*}{Remark}
\newtheorem{ack}{Acknowledgement}

\newcommand{\red}[1]{\textcolor{Red}{#1}}



\begin{document}
\title{Complex Multiplication}
\author[Y-S.~Lee]{Yu-Sheng Lee}
\address{Department of Mathematics, University  of Michigan, Ann Arbor, MI 48109, USA}
\email{yushglee@umich.edu}
\date{\today}

\maketitle
\setcounter{tocdepth}{1}
\tableofcontents


\section{Elliptic curves}

Let $E,E'$ be elliptic curves over a field  $k$,
$\lambda\in \Hom(E,E')$ is called an isogeny if 
\begin{itemize}
	 \item $\lambda\neq 0$
	 \item  $\ker(\lambda)$ is finite
	 \item  $\lambda$ is surjective.
\end{itemize}
$E$ is identified with all its points over the universal domain.
Isogeny is an equivalence relation.

In general,
$\End_\Q(E)=\End(E)\otimes_\Z\Q$ is either 
$\Q$, an imaginary quadratic field, or a quaternion algebra over  $\Q$
ramified at a prime and  $\infty$,
the last situation occurs only when the characteristic of the 
universal domain is  $p$.

Assume that characteristic not  $2,3$, then
 \[
	E:Y^2Z=4X^3-g_2XZ^2-g_3Z^3, \quad
	\Delta=g_2^3-27g_3^2\neq 0,\quad
	j_E=g_2^3/\Delta, J_E=2^{6}3^3j_E.
\]



For an automorphism $\sigma$ of the universal domain, define
 \[
	E^\sigma:y^2=4x^3-g_2^\sigma x-g_3^\sigma,\quad
	j(E^\sigma)=j(E)^\sigma.
\]
when char=0, $\Q(j_E)$ the field of moduli of  $E$,
and  $E$ hs a model defined over which.

\subsection{complex}

Let $L$ be a lattice in  $\C$.
\begin{gather*}
	 \wp(u)=u^{-2}+\sum'_{\omega\in L}[(u-\omega)^{-2}-\omega^{-2}]
	 =u^{-2}+\sum_{n=2}^\infty (2n-1)G_{2n}(L)u^{2n-2},\\
	 \wp'(u)=-2u^{-3}-2\sum'_{\omega\in L}(u-\omega)^{-3}
	 =2u^{-3}+\sum_{n=2}^\infty(2n-1)(2n-2)G_{2n}(L)u^{2n-3},\quad
	 G_{2n}(L)=\sum'_{\omega\in L}\omega^{-2n}.
\end{gather*}
Then
\[
	\wp^{'2}=4\wp^3-g_2(L)\wp-g_3(L),\quad
	g_2(L)=60G_2(L), g_3(L)=140G_3(L).
\]
The function field is $F_L=\C(\wp,\wp')$
and  $g_2(L)^3-27g_3(L)^2\neq0$.

\subsection{torsion}

Let $p$ be the characteristic of  $k$ (possibly  $0$)
and let  $\ell\neq p$ be a prime, 
then  $\bigcup E[\ell^n]\cong (\Ql/\Zl)^2$ and induces
 \[
	 R_\ell\colon \End_\Q(E)\to M_2(\Ql)
\]
then the characteristic polynomial of $R_\ell(\alpha)$,
for  $\alpha\in \End_Q(E)$ is a rational polynomial
not depending on  $\ell$.

There exists Weil pairing
\[
	e_N\colon E[N]\times E[N]\to \mu_N
\]
\begin{itemize}
	\item $e_N(t,s)=e_N(s,t)^{-1}$
	\item  $e_N(s,t)^\sigma=e_N(s^\sigma,t^\sigma)$
	\item if  $t$ is of order  $N$, then  $e_N(t,s)$ is primitive
		for some  $s$.
\end{itemize}

\subsection{isogeny over the complex}
If $\mu\in \C$ and  $\mu L\subset L$, then
 \[
	 \smat{z &\bar{z}\\1 & 1}
	 \smat{\mu &\\ & \mu}=
	 \smat{a&b\\c&d}
	 \smat{z &\bar{z}\\1 & 1}\quad
	 \alpha= \smat{a&b\\c&d}\in M_2(\Z)\cap\GL_2^+(\Q).
\]

\section{main theorem}

Let $E$ be an elliptic curve with CM by  $K$
and  $\theta\colon K\to \End_\Q(E)$ is normalized so that
$\omega\circ \theta(\alpha)=\alpha\omega$ for any diffrential  $\omega$,
here we fixed  $K\hookrightarrow \C$.
Fix $\xi\colon \C/\fa\cong E$, note that 
the image of  $K/\fa$ under  $\xi$
is the torsion points of  $E$.
 \begin{thm}
	 Let $\sigma\in \Aut(\C/K)$, and
	 $s\in \A_K^\times$ such that 
	  $\sigma\vert_{K^{ab}}=\rec(s)$. 
	  Then there is an isomorphism 
	  $\xi'\colon \C/s^{-1}\fa\to E^\sigma$ such that
	   \[
	  	\begin{tikzcd}
			K/\fa \arrow[r,"\xi"]\arrow[d,"s^{-1}"] 
				& E \arrow[d,"\sigma"]\\
			K/s^{-1}\fa \arrow[r,"\xi'"] & E^\sigma
	  	\end{tikzcd}
	  \]
\end{thm}
\begin{proof}
	Proof reduced to when $\fa$ is a fractional ideal in  $K$.
	Let  $h=h(K)$ and $j_1,\cdots,j_h$ be the $j$-invariant.
	$E_i$ defined over  $\Q(j_i)$,
	$\End(E_i)$ defined over  $K(j_i)$.
	Find  $L/K$ sufficiently large so that
	\begin{itemize}
		\item $L$ contains the ray class field of conductor
			$m\cO_K$, where $m\in \Z$
			is such that $\cO_K^\times$
			are dinstinct modulo $m$
		\item $L$ contains all $j_i$
		\item  $L$ contains all  $E[m]$.
	\end{itemize}
	Then fo $\sigma\in \Aut(C/K)$, find a prime ideal
	$\mathfrak{B}$ such that
	\begin{itemize}
		\item $\sigma\vert_L$ is  $\Frob_\mathfrak{B}$
		\item if  $\fp=\mathfrak{B}\cap K$, then
			 $\Nr\fp$ is a rational prime,
			 and  $\fp$ is unramified in  $L$
		 \item $\mathfrak{B}\nmid 6m$
		 \item for every $\tau\in \Gal(L/K)$,
			 $E_i^\tau$ has good reduction mod $\mathfrak{B}$
		 \item The invariants $j_1,\cdots,j_h$
			 are distinct mod  $\mathfrak{B}$
	\end{itemize}
\end{proof}

%\biblio r phystyle{amsalp a}
%\bibliogr phy{biblio}
\end{document}

