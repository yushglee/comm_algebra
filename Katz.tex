\documentclass[leqno]{amsart}
\usepackage{amssymb}
\usepackage{amsmath} 
\usepackage{enumitem}
\usepackage{hyperref}
\usepackage{mathrsfs}
\usepackage{color}
\usepackage{mathtools,caption,bbm,euscript}
\usepackage[table,dvipsnames]{xcolor}
\usepackage{tikz-cd}
\usepackage[utf8]{inputenc}
\usepackage[OT2,T1]{fontenc}
\hypersetup{
 colorlinks=true,
 linkcolor=DarkOrchid,
 filecolor=blue,
 citecolor=olive,
 urlcolor=orange,
 pdftitle={Gauss-Manin},
 %pdfpagemode=FullScreen,
 }
\usepackage{booktabs}

%[label=(\alph*)]
%[label=(\Alph*)]
%[label=(\roman*)]
%[label={(\bfseries R\arabic*)}]


\setlength{\textwidth}{\paperwidth}
\addtolength{\textwidth}{-2in}
\calclayout


\newcommand{\smat}[1]{\left( \begin{smallmatrix} #1 \end{smallmatrix} \right)}
\newcommand{\mat}[1]{\left( \begin{smallmatrix} #1 \end{smallmatrix} \right)}
\newcommand{\dBr}[1]{\llbracket{#1}\rrbracket}
\newcommand{\leg}[2]{\left(\frac{#1}{#2}\right)}

% double bracket
\makeatletter
\newsavebox{\@brx}
\newcommand{\llangle}[1][]{\savebox{\@brx}{\(\m@th{#1\langle}\)}%
  \mathopen{\copy\@brx\kern-0.5\wd\@brx\usebox{\@brx}}}
\newcommand{\rrangle}[1][]{\savebox{\@brx}{\(\m@th{#1\rangle}\)}%
  \mathclose{\copy\@brx\kern-0.5\wd\@brx\usebox{\@brx}}}
  \newcommand{\llbracket}[1][]{\savebox{\@brx}{\(\m@th{#1[}\)}%
  \mathopen{\copy\@brx\kern-0.5\wd\@brx\usebox{\@brx}}}
\newcommand{\rrbracket}[1][]{\savebox{\@brx}{\(\m@th{#1]}\)}%
  \mathclose{\copy\@brx\kern-0.5\wd\@brx\usebox{\@brx}}}
\makeatother

%%% Unitary group specific
\newcommand{\V}{\mathbf V} 
\newcommand{\W}{\mathbf W} 
\newcommand{\G}{\mathbf G} %GU(2,2)
\newcommand{\X}{\mathbf H} %Hermitian symmetric domain
\newcommand{\KK}{\mathbf K} %compact open subgroup
\newcommand{\xx}{\mathbf x}
\newcommand{\yy}{\mathbf y}
\newcommand{\nn}{\mathbf n} %unipotent

\newcommand{\qdr}[1]{\underline{ #1 }}
\newcommand{\cA}{\mathcal A} %complex abelian varieties
\newcommand{\cB}{\mathcal B} %complex abelian varieties
\newcommand{\bB}{\mathbf B}

\newcommand{\mm}{\mathbf{m}}


\newcommand{\balpha}{\boldsymbol{\alpha}} %diagonal torus elements
\newcommand{\bbeta}{\boldsymbol{\beta}} %diagonal torus elements
\newcommand{\bnu}{\boldsymbol{\nu}} %character on diagonal torus
\newcommand{\bmu}{\boldsymbol{\mu}} %character on diagonal torus


\newcommand{\EE}{\EuScript{E}} 




%%% Linear algebraic groups
\DeclareMathOperator{\GL}{GL}
\DeclareMathOperator{\SL}{SL}
\DeclareMathOperator{\SU}{SU}
\DeclareMathOperator{\SO}{SO}
\DeclareMathOperator{\Sp}{Sp}
\DeclareMathOperator{\Mp}{Mp}
\DeclareMathOperator{\GSp}{GSp}
\DeclareMathOperator{\UU}{U}
\DeclareMathOperator{\GUU}{GU}
\DeclareMathOperator{\gl}{\mathfrak{gl}}
\DeclareMathOperator{\mtr}{tr}
\DeclareMathOperator{\diag}{diag}
\DeclareMathOperator{\Ad}{Ad}
\DeclareMathOperator{\adj}{ad}
\DeclareMathOperator{\vol}{vol}

\DeclareMathOperator{\val}{val}
\DeclareMathOperator{\Lie}{Lie}
\DeclareMathOperator{\Pol}{Pol_p}

%%% Adelic rings
\newcommand{\Q}{{\mathbf{Q}}}
\newcommand{\Z}{{\mathbf{Z}}}
\newcommand{\Qp}{\mathbf{Q}_p}
\newcommand{\Zp}{\mathbf{Z}_p}
\newcommand{\Ql}{\mathbf{Q}_\ell}
\newcommand{\Zl}{\mathbf{Z}_\ell}
\newcommand{\R}{\mathbf R}
\newcommand{\C}{\mathbf C}
\newcommand{\A}{\mathbf A}
\newcommand{\hZ}{{\hat{\mathbf{Z}}}}
\newcommand{\dd}{\mathfrak{d}} %different
\newcommand{\DD}{\mathcal{D}}  %discriminant

\newcommand{\arch}{\mathbf{a}}
\newcommand{\fin}{\mathbf{h}}

\DeclareMathOperator{\Sel}{Sel}
\DeclareMathOperator{\Gal}{Gal}
\DeclareMathOperator{\Nr}{\mathsf{N}}
\DeclareMathOperator{\Tr}{Tr}
\newcommand{\qch}{\epsilon} % quadratic character of K/F


%%% Fonts
\newcommand{\oeu}{\EuScript{O}}
\newcommand{\eeu}{\EuScript{E}}
\newcommand{\feu}{\EuScript{F}}
\newcommand{\geu}{\EuScript{G}}
\newcommand{\keu}{\EuScript{K}}

\newcommand{\oo}{\mathcal O}
\newcommand{\bs}{\mathcal S}
\newcommand{\id}{\mathbf{1}}

\newcommand{\1}{\mathbf{1}} 
\newcommand{\bfe}{\mathbf e}
\newcommand{\bff}{\mathbf f}

\newcommand{\bX}{\mathbb{X}}
\newcommand{\bY}{\mathbb{Y}}
\newcommand{\bV}{\mathbb{V}}
\newcommand{\bW}{\mathbb{W}}

\newcommand{\fa}{\mathfrak a}
\newcommand{\fg}{\mathfrak g}
\newcommand{\fc}{\mathfrak c}
\newcommand{\fs}{\mathfrak s}
\newcommand{\fm}{\mathfrak m}
\newcommand{\fn}{\mathfrak n}
\newcommand{\fl}{\mathfrak l}
\newcommand{\fp}{\mathfrak p}
\newcommand{\bfp}{\overline{\mathfrak p}}
\newcommand{\fq}{\mathfrak q}
\newcommand{\bfq}{\overline{\mathfrak q}}
\newcommand{\fk}{\mathfrak k}
\newcommand{\fu}{\mathfrak u}
\newcommand{\ft}{\mathfrak t}

\newcommand{\btheta}{\boldsymbol{\theta}}
\newcommand{\bdelta}{\boldsymbol{\delta}}


\newcommand{\fG}{\mathfrak{G}}
\newcommand{\fX}{\mathfrak{X}}
\newcommand{\euW}{\EuScript{W}}

\DeclareMathOperator{\Der}{Der}

%%% Lie algebras


\newcommand{\rfg}{\mathfrak{g}_0}
\newcommand{\cfg}{\mathfrak{g}}
\newcommand{\rfh}{\mathfrak{h}_0}
\newcommand{\cfh}{\mathfrak{h}}
\newcommand{\rfk}{\mathfrak{k}_0}
\newcommand{\cfk}{\mathfrak{k}}
\newcommand{\rfp}{\mathfrak{p}_0}
\newcommand{\cfp}{\mathfrak{p}}
\newcommand{\rfq}{\mathfrak{q}_0}
\newcommand{\cfq}{\mathfrak{q}}
\newcommand{\rfu}{\mathfrak{u}_0}
\newcommand{\cfu}{\mathfrak{u}}
\newcommand{\rfl}{\mathfrak{l}_0}
\newcommand{\cfl}{\mathfrak{l}}
\newcommand{\rft}{\mathfrak{t}_0}
\newcommand{\cft}{\mathfrak{t}}
\newcommand{\rff}{\mathfrak{f}_0}
\newcommand{\cff}{\mathfrak{f}}

%%% Categorical
\DeclareMathOperator{\Ext}{Ext}
\DeclareMathOperator{\End}{End}
\DeclareMathOperator{\Hom}{Hom}
\DeclareMathOperator{\Inj}{Inj}
\DeclareMathOperator{\Isom}{Isom}
\DeclareMathOperator{\Aut}{Aut}
\DeclareMathOperator{\Ind}{Ind}
\DeclareMathOperator{\coker}{coker}
\DeclareMathOperator{\rank}{rank}
\DeclareMathOperator{\corank}{corank}


\DeclareMathOperator{\Res}{Res}
\DeclareMathOperator{\rec}{rec}



\newtheorem*{theorem*}{Theorem}
\newtheorem{thm}{Theorem}[section]
\newtheorem{lem}[thm]{Lemma}
\newtheorem{prop}[thm]{Proposition}
\newtheorem{cor}[thm]{Corollary}


\theoremstyle{definition}
\newtheorem{definition}[thm]{Definition}
\newtheorem{defn}[thm]{Definition}
\theoremstyle{remark}
\newtheorem{rem}[thm]{Remark}
\newtheorem*{Remark*}{Remark}
\newtheorem{ack}{Acknowledgement}

\newcommand{\red}[1]{\textcolor{Red}{#1}}



\begin{document}
\title{Gauss-Manin}
\author[Y-S.~Lee]{Yu-Sheng Lee}
\address{Department of Mathematics, University  of Michigan, Ann Arbor, MI 48109, USA}
\email{yushglee@umich.edu}
\date{\today}

\maketitle
\setcounter{tocdepth}{1}
\tableofcontents


$U=X\setminus D$ where
$D=\sum_{j=1}^rD_j$ is a reduced normal crossing divisor.
Write $\iota\colon U\hookrightarrow X$ and
\[
	 \Omega_X^a(*D)=\varprojlim_\nu \Omega_X^a(\nu\cdot D)
	 =\iota_*\Omega_U^a
\]
Define $\Omega_X^a(\log D)$ as the subsheaf of $\Omega_X^a(*D)$
consists of sections  $\alpha$ such that 
$\alpha$ and  $d\alpha$ have at most simple poles along  $D$.
\begin{itemize}
	\item $\Omega_X^a(\log D)\hookrightarrow \Omega_X^a(*D)$
	\item  $\Omega_X^a(\log D)=\wedge^a \Omega_X^1(\log D)$
	\item for  $p\in X$ with local parameters
		 $f_1,\cdots,f_n$ such that 
		 $D_j$ is defined by  $f_j=0$.
		 Suppose  $p\in D_j$ for  $j=1,\cdots,s$ and 
		 $p\notin D_j$ for  $j=s+1,\cdots,r$.
		 Let
		 \[
		 	\delta_j=
			\begin{cases}
				\frac{df_j}{f_j} & j\leq s;\\
				df_j & j>s.
			\end{cases}
		 \]
		 then $\Omega^1(\log D)$
		 is locally generated by $\delta_j$
		 over  $\mathcal{O}_X$.
\end{itemize}
\begin{proof}
	Consider $\alpha\in \Omega_X^a(*D)$ written as
	 \[
		\alpha=\alpha_1+\alpha_2\wedge\frac{df_1}{f_1}
	\]
	where $\alpha_i$ are generated by wedge products
	of  $df_2,\cdots,df_n$. 
	Then $\alpha\in \Omega_X^a(\log D)$ implies
	 \[
	f_1\cdot \alpha=f_1\cdot\alpha_1+\alpha_2\wedge df_1,
	\in \Omega_X^a
	f_1\cdot d\alpha=f_1d\alpha_1+d\alpha_2\wedge df_1
	\in \Omega_X^{a+1}
	\]
	so $\alpha_2$ and  $f_1\alpha_1$ have no poles,
	then so does $\alpha_1$ since
	 \[
		 d(f_1\alpha_1)=
		 df_1\wedge\alpha_1+f_1d\alpha-d\alpha_2
	\]
\end{proof}

\begin{align}
	&\alpha\colon \Omega_X^1(\log D)\to
	\bigoplus_{j=1}^s\mathcal{O}_{D_j},\quad
	\alpha(\sum a_j\delta_j)=\bigoplus a_j\vert_{D_j}\\
	&\beta_1\colon \Omega_X^a(\log D)\to 
	\Omega_{D_1}^{a-1}(\log(D-D_1)\vert_{D_1})\quad
	\beta_1(\varphi_1+\varphi_2\wedge\frac{df_1}{f_1})
	=\sum a_I\delta_{I-\{1\}}\vert_{D_1},
	\varphi_2=\sum a_I\delta_{I-\{1\}},
	\varphi_1=\sum_{1\notin I} a_I\delta_I.\\
	&\gamma_1\colon 
	\Omega_X^a(\log(D-D_1))\to 
	\Omega_{D_1}^a(\log(D-D_1)\vert_{D_1}),\quad
	\gamma_1(\sum_{1\in I}f_1a_I\delta_I+
	\sum_{1\notin I}a_I\delta_I)=
	\sum_{1\notin I}a_I\delta_I\vert_{D_1}
\end{align}

\begin{gather*}
	0\to \Omega_X^1\to \Omega_X^11(\log D)\to 
	\bigoplus_{j=1}^r\mathcal{O}_{D_j}\to 0\\
	0\to \Omega_X^a(\log(D-D_1))\to 
	\Omega_X^a(\log D)\xrightarrow{\beta}
	\Omega_{D_1}^{a-1}(\log(D-D_1)\vert_{D_1})\to 0\\
	0\to \Omega_X^a(\log D)(-D_1)\to
	\Omega_X^a(\log(D-D_1))\xrightarrow{\gamma}
	\Omega_{D_1}^a(\log(D-D_1)\vert_{D_1})\to 0
\end{gather*}

\begin{defn}
	From 
	\[
		\nabla\colon E\to \Omega_X^1(\log D)\otimes E,\quad
		\nabla(f\cdot e)=f\cdot \nabla(e)+df\otimes e
	\]
	define 
	\[
		\nabla_a\colon \Omega_X^a(\log D)\otimes E
		\to \Omega_X^{a+1}(\log D)\otimes E,\quad
		\nabla_a(\omega\otimes e)=d\omega\otimes e+
		(-1)^a\cdot \omega\wedge\nabla(e)
	\]
\end{defn}

\section{Relative de Rham complex}
$f\colon U\to V$ proper smooth morphism between
smooth  $\C$-schemes
and  $(M,\nabla)$ an algebraic differential equation on  $U$,
i.e. $M$ is locally free sheaf on  $U$ and integrable connection
 \[
	 \nabla\colon M\to \Omega_U^1\otimes_{\mathcal{O}_U}M
\]
compose with $\Omega_U^1\to \Omega_{U/V}^1$,
get relative de Rham complex, whose hypercohomology.
\[
	H^q_{dR}(U/V,(M,\nabla))=
	\mathbb{R}^qf_*(\Omega_{U/V}^*\otimes_{\mathcal{O}_U}M)
\]

Filter the absolute de Rham complex by
\[
	F^i(\Omega_U^*\otimes_{\mathcal{O}_U}M)=
	\text{image of }
	f^*(\Omega_V^i)\otimes_{\mathcal{O}_U}
	\Omega_U^{*-i}\otimes_{\mathcal{O}_U}M
	\to \Omega_U^*\otimes_{\mathcal{O}_U}M\quad
	gr^i=F^i/F^{i+1}=
	f^*(\Omega_V^i)\otimes_{\mathcal{O}_U}
	\Omega_U^{*-i}\otimes_{\mathcal{O}_U}M
\]
The Gauss-Manin connection is the coboundary map of the 
long exact sequence of $\mathbb{R}^qf_*$
from 
 $0\to gr^1\to F^0/F^2\to gr^0\to 0$, since
  \begin{gather*}
	  \mathbb{R}^qf_*(gr^0)=H^q_{dR}(U/V,(M,\nabla))\\
	  \mathbb{R}^{q+1}f_*(gr^1)=
	  \Omega_V^1\otimes_{\mathcal{O}_V}H^q_{dR}(U/V,(M,\nabla))
 \end{gather*}
 (why integrable?)
 and thus $H^q_{dR}(U/V,(M,\nabla))$
 are locally free as well.

 When $(M,\nabla)=(\mathcal{O}_U,d)$,
 then  $H_{dR}(U/V)=\mathbb{R}f_*(\Omega_{U/V}^*)$
 is the Gauss-Manin connection
 and the algebraic differntial equations is called
 the Picard-Fuchs equation.

\[
	\begin{tikzcd}
		U\arrow[r,hookrightarrow]
		 \arrow[d,"f"]
		 & S \arrow[d,"\pi"]\\
		V\arrow[r,hookrightarrow] & T
	\end{tikzcd}
\]
where $V=T\setminus Y$ and  $D=\pi^{-1}(Y)$ is normal crossing.
Define the locally free sheaf 
of relative differrntials with logarithmic singularities along  $D$
by 
 \[
	 \Omega_{S/T}^1(\log D)=
	 \Omega_S^1(\log D)/\pi^*(\Omega_T^1(\log Y)),
	 \Omega_{S/T}^p(\log D)=
	 \wedge_{\mathcal{O}_S}^p\Omega_{S/T}^1(\log D)=
\]
which fits into
\[
	0\to\pi^*(\Omega_T^1(\log Y))\otimes_{\mathcal{O}_S}
	\Omega_{S/T}^{*-1}(\log D)\to 
	\Omega_S^*(\log D)\to 
	\Omega_{S/T}^*(\log )\to 0
\]

When $(M,\nabla)$ has regular singular points, i.e.
there is locally free  $\overline{M}$ on  $S$ which prolongs  $M$
and 
 \[
	 \bar{\nabla}\colon \overline{M}
	 \to \Omega_S^1(\log D)\otimes_{\mathcal{O}}\bar{M}
\]
Filter $\Omega_S^*(\log D)\otimes_{\mathcal{O}}\bar{M}$
by subcomplexes
 \[
	F^i=\text{image of }
	\pi^*(\Omega_T^i(\log Y)\otimes_{\mathcal{O}_S}
	\Omega_S^{*-i}(\log D)\otimes_{\mathcal{O}_S}\bar{M}
	\to \Omega_S^*(\log D)\otimes_{\mathcal{O}_S}\bar{M}
\]
then
$gr^i=F^i/F^{i+1}=\pi^*(\Omega_T^i(\log Y)\otimes_{\mathcal{O}_S}
	\Omega_S^{*-i}(\log D)\otimes_{\mathcal{O}_S}\bar{M}$
and $gr^0$ prolongs  $\Omega_{U/V}\otimes M$.
Same constructions provide prolongations.
So the relative de Rham coholomologies on $V$
has regular singular points.

\subsection{exponents}
The map
\[
	L_i\colon \overline{M}
	\to \Omega_S^1(\log D)\otimes_{\mathcal{O}_S}\overline{M}
	\xrightarrow{\text{residue}}
	\mathcal{O}_{D_i}\otimes_{\mathcal{O}_S}\overline{M}
\]
is $\mathcal{O}_{D_i}$-linear.
Since $D_i$ is proper, 
the characteristic polynomial
$P_i=\det(XI-L_i;\overline{M}\otimes\mathcal{O}_{D_i})\in \C[X]$ 
whose roots are called the exponents
of $(\overline{M},\bar{\nabla})$ around $D_i$

\section{Katz-Oda}

$S$ is smooth scheme over the field  $k$,
and $\EE$ is a quasi-coherent sheaf of  $\oo_S$-module.
Let  $\Der_k(\oo_S)$ be the sheaf of $k$-derivations of 
 $\oo_S$ into itself,
which is naturally a sheaf of  $k$-Lie algbras,
whiles as  $\oo_S$-module isomorphic to
$\underline{\Hom}_{\oo_S}(\Omega_{S/k}^1,\oo_S)$.
Let $\underline{\End}_k(\EE)$
be the sheaf of  $k$-linear endomorhism.

Given a connection on $\EE$,
there is a  $\oo_S$-linear map
 \[
	 \Der_k(\oo_S)\to \underline{\End}_k(\EE),\quad
	 D\mapsto
	 \tilde{D}\colon 
	 \EE\to \Omega_{S/k}^1\otimes_{\oo_S}\EE
	 \xrightarrow{D\otimes 1}
	 \oo_S\otimes_{\oo_S}\EE\cong \EE
\]

Conversely, since $S$ is smooth over  $k$,
any  $\oo_S$-linear map
$\Der_k(\oo_S)\to \underline{\End}_k(\EE)$
with  $ \tilde{D}(fe)=D(f)e+f\tilde{D}(e)$
comes from a unique connection.
The connection is integrable iff
the above map is a Lie-algebra homomorphism.

 \[
	 d_1^{p,q}\colon E_1^{p,q}=R^{p+q}\pi_*(gr^p)\to 
	 E_1^{p+1,q}=R^{p+1+q}\pi_*(gr^{p+1})
\]
is the connecting homomorphism of 
\[
	0\to gr^{p+1}\to F^p/F^{p+2}\to gr^p\to 0.
\]

local system $\EE$ and 
integrable connection on a coherent sheaves $\EE\otimes\oo_X$

\section{Vogan-Zuckerman classification}

Let $G$ be a connected real semisimple Lie group
with finite center,
$\rfg$ be the Lie algebra
and  $\cfg$ be the complexification.
Denote by
\[
	\langle X,Y\rangle=\Tr(\adj(X)\adj(Y))
\]
the Killing form.
Fix a Cartan involution $\Theta$ on  $G$
and  $\theta$ on  $\cfg$.
Then $K=G^\Theta$ is a maximal compact subgroup
(used that the center is finite).
Let  $\rfg=\rfk\oplus \rfp$
be the Cartan decomposition
( $\pm1$-eigenspaces of  $\theta$)

 \begin{defn}
	 Let $(\pi,\mathcal{H})$
	 be an irreducible unitary 
	 representation of $G$.
	 Then the smooth vectors
	 $\mathcal{H}^\infty\subset \mathcal{H}$
	  is a dense subset that is 
	  invariant under $G$,
	  and the  $K$-finite subspace
	  $\mathcal{H}^K\subset\mathcal{H}^\infty$
	  is an irreducible  $(\cfg, K)$-module
	  called the Harish-Chandra module
	  attached to  $\pi$.
\end{defn}

\begin{defn}
	Define $\theta$-stable parabolic
	subalgebra as follows.
	Let $x\in i\rfk$, then
	$\adj(x)$ acts on  $G$
	is diagonalization with real 
	eigenvalues, let
	\begin{align*}
		\cfq&\colon \text{non-negative eigenspace}\\
		\cfu&\colon \text{positive eigenspace}\\
		\cfl&\colon \text{zero-negative eigenspace}
	\end{align*}
        \begin{enumerate}[label=(\alph*)]
		\item $\cfq$ is a parabolic subalgebra and
		$\cfq=\cfl+\cfu$
		is the Levi decomposition.
		\item $\cfl$ is the complexification 
		of $\rfl=\cfq\cap \rfg$.
		\item $\rfl$ correspondes to 
		the connected closed subgroup $L\subset G$
		which is also the centralizer of  $x$.
		\item $\cfq=\cfq\cap\cfk+\cfq\cap\cfp$
		\item $\cfq\cap \cfk$ 
		is a parabolic subalgebra of $\cfk$ and
		$\cfq\cap\cfk=\cfl\cap\cfk+\cfu\cap\cfk$
		is the Levi decomposition.
       \end{enumerate}
       Pick a Cartan subalgebra 
       $\rft$ of  $\rfk$ that contains  $ix$.
       Then  $\cft\subset \cfl\cap\cfk$
       If $\cff\subset\cfq$ is  $\adj(\cft)$-stable, define
        \[
		\Delta(\cff,\cft)=\{\alpha\in \cft^*\mid 
		\text{ roots with multiplicities}\},\quad
		\rho(\cff)=\frac{1}{2}\sum_{\alpha\in \Delta(\cff)}
		\alpha\in \cft^*,
		\rho(\cff)(y)=\frac{1}{2}\Tr(\adj(y)\vert_{\cff})
		\text{ for }y\in \cft.
       \]
\end{defn}
A representation $\lambda\colon \cfl\to \C$ is called admissible
if 
\begin{itemize}
	\item it is the differential of a unitary representation
		$\lambda\colon L\to \C^\times$.
	\item if  $\alpha\in \Delta(\cfu)$, then
	$ \langle \alpha, \lambda\vert_{\cft}\rangle\geq 0$
\end{itemize}
\begin{defn}
	If $\lambda$ is admissible, let 
	$\mu(\cfq,\lambda)$ be the irreducible representation
	of  $K$ of highest weight  
	$\lambda\vert_{\cft}+2\rho(\cfu\cap \cfp)$.
	Simply write $\mu(\cfq,\lambda)=\mu(\cfq)$
	when  $\lambda=0$ correspondes to the trivial representaton.
\end{defn}
\begin{thm}
	There exists a unique irreducible $\cfg$--module
	$A_\cfq(\lambda)$ with the following properties.
	  \begin{enumerate}[label=(\alph*)]
	 	\item the restriction to $\cfk$
			contains $\mu(\cfq,\lambda)$
		\item the infinitesimal character is 
			$\chi_{\lambda+\rho}$
		\item if a  $K$-representation of highest weight
			$\delta$ occurs in  $A_\cfq(\lambda)$
			then 
			 \[
				\delta=\lambda\vert_{\cft}
				+2\rho(\cfu\cap\cfp)
			+\sum_{\beta\in \Delta(\cfu\cap\cfp)}
				n_\beta\beta,\quad n_\beta\geq 0
			\]
	 \end{enumerate}
\end{thm}
\begin{itemize}
	\item $\cfl\subset\cfk$ if and only if $A_\cfq(\lambda)$
		is a discrete series.
	\item  $[\cfl,\cfl]\subset\cfk$ if and only if
		$A_\cfq(\lambda)$ is a fundamental series 
		representation(?) 
	\item  $[\cfl,\cfl]\nsubseteq\cfk$ then
		$A_\cfq(\lambda)$ is non-tempered.
	\item if  $\cfu\cap\cfp=0$ 
		(for example when  $\cfq=\cfg)$,
		then  $A_\cfq(\lambda)=\lambda$.
		(at least when $\lambda$ is trivial).
\end{itemize}

Let $\Lambda^*\cfp$ be the quotient of   $\Lambda^*\cfg$
by the ideal generated by  $\cfk$.
Then for an $\cfg$-module  $X$,
$\Hom(\Lambda^*\cfp,X)\subset \Hom(\Lambda^*\cfg,X)$
and the differential preserves $\Hom_\cfk(\Lambda^*\cfp,X)$
and defines  $H^*(\cfg,\cfk,X)$.

\begin{thm}
	Let $F$ be a finite-dimensional $G$-representation
	and  $\mathcal{H}$ an irreducible unitary representation.
	\begin{enumerate}[label=(\alph*)]
		\item $H^*(G,\mathcal{H}\otimes F)\cong 
			H^*(\cfg,\cfk,\mathcal{H}^K\otimes F)$.
		\item if the actions of the Casimir operator
			on $\mathcal{H}^K$ and $F$
			don't agree, then
			$H^*(\cfg,\cfk,\mathcal{H}^K\otimes F)=0$
		\item if they agree, then
			 $H^*(\cfg,\cfk,\mathcal{H}^K\otimes F)=
		  Hom_\cfk(\Lambda^*\cfp,\mathcal{H}^K\otimes F)$.
	\end{enumerate}
\end{thm}

\begin{thm}
	Let $R=\dim(\cfu\cap\cfp)$
	and $F$ be a f nite-dimensional irreducible $\cfg$-module
	of lowest weight $-\gamma\in \cfh^*$
	with respect to  $\Delta^+(\cfg,\cfh)$,
	then if $\gamma=\lambda\vert_{\cfh}$
	\[
		H^\iota(\cfg,\cfk,A_\cfq(\lambda)\otimes F)\cong
		H^{\iota-R}(\cfl,\cfk\cap\cfl,\C)\cong
		\Hom_{\cfl\cap\cfk}
		(\Lambda^{\iota-R}(\cfl\cap\cfp), \C)
	\]
	and otherwise
	$H^\iota(\cfg,\cfk,A_\cfq(\lambda)\otimes F)=0$.
\end{thm}

\begin{thm}
	Suppose $H^*(G,\pi\otimes F)\neq 0$,
	then there exists  $\cfq=\cfl+\cfu$ such that
	 \begin{enumerate}[label=(\alph*)]
		\item $F/\cfu F$ is a one-dimensional 
			unitary representation $\lambda$ of $L$.
			Let $-\lambda\colon \cfl\to \C$
			be the differential of which.
		\item  $X\cong A_\cfq(\lambda)$
	\end{enumerate}
\end{thm}

\begin{thm}
	When $G$ is simple and  $\pi$ is a nontrivial
	unitary irreducible representation
	and  $E$ is finite dimensional.
	Then  $H^i(G,\pi\otimes E)=0$ 
	for all $i<r_G$, which is the minimum dimension
	of  $\cfu\cap \cfp$.
\end{thm}
\begin{align*}
	\SL(2,\R)\colon & r_G=1\\
	\SU(p,q),p\leq q\colon& r_G=p\\
	\SO(p,q),p\leq q,2\leq q\colon& r_G=p.
\end{align*}

\subsection{Hermitian symmetric}

$\Lambda^\iota\cfp=\oplus_{p+q=\iota}(\Lambda^p\cfp^+)
\otimes(\Lambda^q\cfp^-)$ and
\[
	\Hom_\cfk(\Lambda^\iota\cfp,X)\cong
	\bigoplus_{p+q=\iota}
	\Hom_\cfk((\Lambda^p\cfp^+) \otimes(\Lambda^q\cfp^-),X)
\]
with $d=\partial+\bar{\partial}$.
Thus 
$H^i(\cfg,\cfk,X)\cong \oplus_{p+q=i}H^{p,q}(\cfg,\cfk,X)$

\begin{prop}
	Let $R^{\pm}=\dim(\cfu\cap\cfp^\pm)$, 
	then
	\[
		H^{p+R^+,p+R^-}(\cfg,\cfk,A_\cfq(\lambda)\otimesF)
		\cong H^{p,p}(\cfl,\cfl\cap\cfk,\C)
		\cong \Hom_{\cfl\cap\cfk}
		(\Lambda^{2p}(\cfl\cap\cfp),\C).
	\]
	the cohomology $H^{p,q}=0$ 
	if $p-q\neq R^+-R^-$.
\end{prop}


%\bib ingraphystyle{amsalpha}
%\bib raphy{biblio}
\end{document}

