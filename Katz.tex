\documentclass[leqno]{amsart}
\usepackage{amssymb}
\usepackage{amsmath} 
\usepackage{enumitem}
\usepackage{hyperref}
\usepackage{mathrsfs}
\usepackage{color}
\usepackage{mathtools,caption,bbm,euscript}
\usepackage[table,dvipsnames]{xcolor}
\usepackage{tikz-cd}
\usepackage[utf8]{inputenc}
\usepackage[OT2,T1]{fontenc}
\hypersetup{
 colorlinks=true,
 linkcolor=DarkOrchid,
 filecolor=blue,
 citecolor=olive,
 urlcolor=orange,
 pdftitle={Gauss-Manin},
 %pdfpagemode=FullScreen,
 }
\usepackage{booktabs}

%[label=(\alph*)]
%[label=(\Alph*)]
%[label=(\roman*)]
%[label={(\bfseries R\arabic*)}]


\setlength{\textwidth}{\paperwidth}
\addtolength{\textwidth}{-2in}
\calclayout


\newcommand{\smat}[1]{\left( \begin{smallmatrix} #1 \end{smallmatrix} \right)}
\newcommand{\mat}[1]{\left( \begin{smallmatrix} #1 \end{smallmatrix} \right)}
\newcommand{\dBr}[1]{\llbracket{#1}\rrbracket}
\newcommand{\leg}[2]{\left(\frac{#1}{#2}\right)}

% double bracket
\makeatletter
\newsavebox{\@brx}
\newcommand{\llangle}[1][]{\savebox{\@brx}{\(\m@th{#1\langle}\)}%
  \mathopen{\copy\@brx\kern-0.5\wd\@brx\usebox{\@brx}}}
\newcommand{\rrangle}[1][]{\savebox{\@brx}{\(\m@th{#1\rangle}\)}%
  \mathclose{\copy\@brx\kern-0.5\wd\@brx\usebox{\@brx}}}
  \newcommand{\llbracket}[1][]{\savebox{\@brx}{\(\m@th{#1[}\)}%
  \mathopen{\copy\@brx\kern-0.5\wd\@brx\usebox{\@brx}}}
\newcommand{\rrbracket}[1][]{\savebox{\@brx}{\(\m@th{#1]}\)}%
  \mathclose{\copy\@brx\kern-0.5\wd\@brx\usebox{\@brx}}}
\makeatother

%%% Unitary group specific
\newcommand{\V}{\mathbf V} 
\newcommand{\W}{\mathbf W} 
\newcommand{\G}{\mathbf G} %GU(2,2)
\newcommand{\X}{\mathbf H} %Hermitian symmetric domain
\newcommand{\KK}{\mathbf K} %compact open subgroup
\newcommand{\xx}{\mathbf x}
\newcommand{\yy}{\mathbf y}
\newcommand{\nn}{\mathbf n} %unipotent

\newcommand{\qdr}[1]{\underline{ #1 }}
\newcommand{\cA}{\mathcal A} %complex abelian varieties
\newcommand{\cB}{\mathcal B} %complex abelian varieties
\newcommand{\bB}{\mathbf B}

\newcommand{\mm}{\mathbf{m}}


\newcommand{\balpha}{\boldsymbol{\alpha}} %diagonal torus elements
\newcommand{\bbeta}{\boldsymbol{\beta}} %diagonal torus elements
\newcommand{\bnu}{\boldsymbol{\nu}} %character on diagonal torus
\newcommand{\bmu}{\boldsymbol{\mu}} %character on diagonal torus


\newcommand{\EE}{\EuScript{E}} 




%%% Linear algebraic groups
\DeclareMathOperator{\GL}{GL}
\DeclareMathOperator{\SL}{SL}
\DeclareMathOperator{\SU}{SU}
\DeclareMathOperator{\SO}{SO}
\DeclareMathOperator{\Sp}{Sp}
\DeclareMathOperator{\Mp}{Mp}
\DeclareMathOperator{\GSp}{GSp}
\DeclareMathOperator{\UU}{U}
\DeclareMathOperator{\GUU}{GU}
\DeclareMathOperator{\gl}{\mathfrak{gl}}
\DeclareMathOperator{\sll}{\mathfrak{sl}}
\DeclareMathOperator{\spp}{\mathfrak{sp}}
\DeclareMathOperator{\mtr}{tr}
\DeclareMathOperator{\diag}{diag}
\DeclareMathOperator{\Ad}{Ad}
\DeclareMathOperator{\adj}{ad}
\DeclareMathOperator{\vol}{vol}

\DeclareMathOperator{\val}{val}
\DeclareMathOperator{\Lie}{Lie}
\DeclareMathOperator{\Pol}{Pol_p}

%%% Adelic rings
\newcommand{\Q}{{\mathbf{Q}}}
\newcommand{\Z}{{\mathbf{Z}}}
\newcommand{\Qp}{\mathbf{Q}_p}
\newcommand{\Zp}{\mathbf{Z}_p}
\newcommand{\Ql}{\mathbf{Q}_\ell}
\newcommand{\Zl}{\mathbf{Z}_\ell}
\newcommand{\R}{\mathbf R}
\newcommand{\C}{\mathbf C}
\newcommand{\A}{\mathbf A}
\newcommand{\hZ}{{\hat{\mathbf{Z}}}}
\newcommand{\dd}{\mathfrak{d}} %different
\newcommand{\DD}{\mathcal{D}}  %discriminant

\newcommand{\arch}{\mathbf{a}}
\newcommand{\fin}{\mathbf{h}}

\DeclareMathOperator{\Sel}{Sel}
\DeclareMathOperator{\Gal}{Gal}
\DeclareMathOperator{\Nr}{\mathsf{N}}
\DeclareMathOperator{\Tr}{Tr}
\newcommand{\qch}{\epsilon} % quadratic character of K/F


%%% Fonts
\newcommand{\oeu}{\EuScript{O}}
\newcommand{\eeu}{\EuScript{E}}
\newcommand{\feu}{\EuScript{F}}
\newcommand{\geu}{\EuScript{G}}
\newcommand{\keu}{\EuScript{K}}

\newcommand{\oo}{\mathcal O}
\newcommand{\bs}{\mathcal S}
\newcommand{\id}{\mathbf{1}}

\newcommand{\1}{\mathbf{1}} 
\newcommand{\bfe}{\mathbf e}
\newcommand{\bff}{\mathbf f}

\newcommand{\bX}{\mathbb{X}}
\newcommand{\bY}{\mathbb{Y}}
\newcommand{\bV}{\mathbb{V}}
\newcommand{\bW}{\mathbb{W}}

\newcommand{\fa}{\mathfrak a}
\newcommand{\fg}{\mathfrak g}
\newcommand{\fc}{\mathfrak c}
\newcommand{\fs}{\mathfrak s}
\newcommand{\fm}{\mathfrak m}
\newcommand{\fn}{\mathfrak n}
\newcommand{\fl}{\mathfrak l}
\newcommand{\fp}{\mathfrak p}
\newcommand{\bfp}{\overline{\mathfrak p}}
\newcommand{\fq}{\mathfrak q}
\newcommand{\bfq}{\overline{\mathfrak q}}
\newcommand{\fk}{\mathfrak k}
\newcommand{\fu}{\mathfrak u}
\newcommand{\ft}{\mathfrak t}

\newcommand{\btheta}{\boldsymbol{\theta}}
\newcommand{\bdelta}{\boldsymbol{\delta}}


\newcommand{\fG}{\mathfrak{G}}
\newcommand{\fX}{\mathfrak{X}}
\newcommand{\euW}{\EuScript{W}}

\DeclareMathOperator{\Der}{Der}

%%% Lie algebras


\newcommand{\rfg}{\mathfrak{g}_0}
\newcommand{\cfg}{\mathfrak{g}}
\newcommand{\rfh}{\mathfrak{h}_0}
\newcommand{\cfh}{\mathfrak{h}}
\newcommand{\rfk}{\mathfrak{k}_0}
\newcommand{\cfk}{\mathfrak{k}}
\newcommand{\rfp}{\mathfrak{p}_0}
\newcommand{\cfp}{\mathfrak{p}}
\newcommand{\rfq}{\mathfrak{q}_0}
\newcommand{\cfq}{\mathfrak{q}}
\newcommand{\rfu}{\mathfrak{u}_0}
\newcommand{\cfu}{\mathfrak{u}}
\newcommand{\rfl}{\mathfrak{l}_0}
\newcommand{\cfl}{\mathfrak{l}}
\newcommand{\rft}{\mathfrak{t}_0}
\newcommand{\cft}{\mathfrak{t}}
\newcommand{\rff}{\mathfrak{f}_0}
\newcommand{\cff}{\mathfrak{f}}
\newcommand{\rt}{\Delta}

%%% Categorical
\DeclareMathOperator{\Ext}{Ext}
\DeclareMathOperator{\End}{End}
\DeclareMathOperator{\Hom}{Hom}
\DeclareMathOperator{\Inj}{Inj}
\DeclareMathOperator{\Isom}{Isom}
\DeclareMathOperator{\Aut}{Aut}
\DeclareMathOperator{\Ind}{Ind}
\DeclareMathOperator{\coker}{coker}
\DeclareMathOperator{\rank}{rank}
\DeclareMathOperator{\corank}{corank}


\DeclareMathOperator{\Res}{Res}
\DeclareMathOperator{\rec}{rec}
\DeclareMathOperator{\Sym}{Sym}



\newtheorem*{theorem*}{Theorem}
\newtheorem{thm}{Theorem}[section]
\newtheorem{lem}[thm]{Lemma}
\newtheorem{prop}[thm]{Proposition}
\newtheorem{cor}[thm]{Corollary}


\theoremstyle{definition}
\newtheorem{definition}[thm]{Definition}
\newtheorem{defn}[thm]{Definition}
\theoremstyle{remark}
\newtheorem{rem}[thm]{Remark}
\newtheorem{example}[thm]{Example}
\newtheorem*{Remark*}{Remark}
\newtheorem{ack}{Acknowledgement}

\newcommand{\red}[1]{\textcolor{Red}{#1}}



\begin{document}
\title{Gauss-Manin}
\author[Y-S.~Lee]{Yu-Sheng Lee}
\address{Department of Mathematics, University  of Michigan, Ann Arbor, MI 48109, USA}
\email{yushglee@umich.edu}
\date{\today}

\maketitle
\setcounter{tocdepth}{1}
\tableofcontents


Recall that for a complex semisimple Lie algebra $\cfg$,
the Killing form  $B(X,Y)=\Tr(\adj(X)\adj(Y))$
is a non-degenerate symmetric pairing on
the Cartan subalgebra $\cfh$.
Let $\rt\subset\cfh^*$ be the root system.
\begin{itemize}
	\item $B(H,H')=\sum_{\alpha\in\rt}\alpha(H)\alpha(H')$
		for $H,H'\in \cfh$.
	\item for $\alpha\in\rt$ define $H_\alpha\in\cfh$ by
		$\alpha(H)=B(H,H_\alpha)$.
	\item for $\alpha\in\rt$ one may choose
		$E_\alpha\in \cfg_\alpha$
		such that $B(E_\alpha,E_{-\alpha})=1$.
\end{itemize}
Let $\langle\cdot,\cdot\rangle$
be the pairing on $\cfh^*$ induced by
$\langle\alpha,\beta\rangle=B(H_\alpha,H_\beta)$.
\begin{itemize}
	\item $2\frac{\langle\beta,\alpha\rangle}{|\alpha|^2}
		\in\{0,\pm1,\pm2,\pm3,\pm4\}$
		for any $\alpha,\beta\in\rt$.
	\item  $\lambda\in \cfh^*$ is said to be
		algebraically integral if 
		$2\frac{\langle\lambda,\alpha\rangle}
		{|\alpha|^2}\in\Z$
		for all $\alpha\in\rt$.
	\item $\lambda\in \cfh^*$ is said to be
		dominant with respect to  $\rt^+$
		if $\langle\lambda,\alpha\rangle\geq0$
		for all $\alpha\in\rt^+$.
\end{itemize}
Then given a dominant and algebraically integral
$\lambda\in\cfh^*$,
there exists uniquely an irreducible finite-dimensional 
$\cfg$-representation  $V_\lambda$ of highest weight $\lambda$.


\subsection{infinitesimal character}
The universal enveloping algebra
$\mathcal{U}(\cfg)$
admits a decomposition
\[
	\mathcal{U}(\cfg)=
	\mathcal{U}(\cfh)\bigoplus 
	\sum_{\alpha\in\Delta^+}E_{-\alpha}\mathcal{U}(\cfg).
\]
Denote by $\tau'$
the projection of $Z(\mathcal{U}(\cfg))$ to $\mathcal{U}(\cfh)$
and by $\tau\colon \mathcal{U}(\cfh)\to\mathcal{U}(\cfh)$ 
the homomorphism induced by $H\to H-\delta(H)$.
Then the Harish-Chandra isomorphism is defined as
\[
	\gamma=\tau\circ\gamma'\colon Z(\mathcal{U}(\cfg))\to
	\mathcal{U}(\cfh)^W
\]
For any $\lambda\in\cfh^*$,
denote by $\chi_\lambda\colon Z(\mathcal{U}(\cfg))\to\C^\times$
the character given by composing 
$\lambda\colon \mathcal{U}(\cfh)\to \C^\times$
with the $\gamma$.

If $\lambda\in\cfh^*$ is dominant and algebraically integral,
then the infinitesimal character on  $V_\lambda$
is  $\chi_{\lambda+\rho}$.

\begin{example}
	When $\cfg=\sll_2$,
	the center  $Z(\mathcal{U}(\cfg))$
	is generated by the Casimir operator
	$\Omega=\frac{1}{2}h^2+ef+fe$,
	\[
		\frac{1}{2}h^2+ef+fe=(\frac{1}{2}h^2+h)+2fe
		\xrightarrow{\gamma'} \frac{1}{2}h^2+h
		\xrightarrow{\tau} \frac{1}{2}(h-1)^2+(h-1)
		=\frac{1}{2}h^2-\frac{1}{2}
	\]
	Write $\rt=\{\pm\alpha\}$,
	where $\alpha(h)=2$.
	Then $\lambda=\lambda(\alpha)$ identifies
	the set of algebraically integral weights
	with $\Z$
	and $\lambda$ is dominant if and only if  $\lambda\geq 0$.
	Then $V_\lambda=Sym^\lambda\C^2$
	for such $\lambda$,
	and  $\Omega$ acts by  $\frac{1}{2}\lambda^2+\lambda$
	on which.
\end{example}

\subsection{Vogan-Zuckerman classification}

Let $G$ be a connected semisimple real Lie group
with finite center,
$\rfg$ be the Lie algebra,
and $\cfg$ be the complexification.
Fix a Cartan involution $\theta$
and let  $\rfg=\rfk\oplus\rfp$
be the Cartan decomposition.
Then $\rfk$ is the Lie algebra of 
$K=G^\theta$, which is a maximal compact subgroup
thanks to the finiteness of the center.
\begin{defn}
	Pick $x\in i\rfk$,
	then the action of $\adj(x)$ on $\cfg$
	is diagonalizable with real eigenvalues. Put
	\begin{align*}
		\cfq&\colon \text{sum of non-negative eigenspace}\\
		\cfu&\colon \text{sum of positive eigenspace}\\
		\cfl&\colon \text{sum of zero-negative eigenspace}
	\end{align*}
        \begin{itemize}
		\item $\cfq$ is a parabolic subalgebra and
		$\cfq=\cfl+\cfu$
		is the Levi decomposition.
		\item $\cfl$ is the complexification 
		of $\rfl=\cfq\cap \rfg$.
		\item $\cfq=\cfq\cap\cfk+\cfq\cap\cfp$
		\item $\cfq\cap \cfk$ 
		is a parabolic subalgebra of $\cfk$ and
		$\cfq\cap\cfk=\cfl\cap\cfk+\cfu\cap\cfk$
		is the Levi decomposition.
       \end{itemize}
       Such $\cfq$ is called a  $\theta$-stable
       parabolic subalgebra.
       Let $\rft$ be a Cartan subalgebra 
       of $\rfk$ that contains $ix$,
       then $\cft\subset \cfl\cap\cfk$.
       For any $\adj(\cft)$-stable subspace $\cff\subset\cfq$,
       let $\rt(\cff)\subset \cft^*$
       be the roots space, counted with multiplicities,
       and define 
       \[
		\rho(\cff)=\frac{1}{2}\sum_{\alpha\in \Delta(\cff)}
		\alpha\in \cft^*,\quad
		\rho(\cff)(y)=\frac{1}{2}\Tr(\adj(y)\vert_{\cff})
		\text{ for }y\in \cft.
       \]

	Let $L\subset G$ be the closed subgroup
	corresponding to $\rfl$,
	which is also the centralizer of  $x$.
	A representation $\lambda\colon \cfl\to \C$ 
	is called admissible if 
	it is the differential of a unitary character
	$\lambda\colon L\to \C^\times$ and
	\[
	\langle \alpha, \lambda\vert_{\cft}\rangle\geq 0	
	\text{ for all }\alpha\in\rt(\cfu).
	\]
	When this is the case, denote by
	$\mu(\cfq,\lambda)$ the irreducible representation
	of $K$ of highest weight  
	$\lambda\vert_{\cft}+2\rho(\cfu\cap \cfp)$.
\end{defn}

\begin{thm}
	There exists uniquely an irreducible $\cfg$-module
	$A_\cfq(\lambda)$ with the following properties.
	  \begin{enumerate}[label=(\alph*)]
		  \item the restriction of $A_\cfq(\lambda)$
			to $\cfk$ contains $\mu(\cfq,\lambda)$.
		\item the infinitesimal character of 
			$A_\cfq(\lambda)$ is $\chi_{\lambda+\rho}$.
		\item if an irreducible $K$-representation of 
			highest weight $\delta$ occurs in
			$A_\cfq(\lambda)$, then 
			 \[
				\delta=\lambda\vert_{\cft}
				+2\rho(\cfu\cap\cfp)
			+\sum_{\beta\in \Delta(\cfu\cap\cfp)}
				n_\beta\beta,\quad n_\beta\geq 0.
			\]
	 \end{enumerate}
\end{thm}
\begin{itemize}
	\item $\cfl\subset\cfk$ if and only if $A_\cfq(\lambda)$
		is a discrete series.
	\item  $[\cfl,\cfl]\subset\cfk$ if and only if
		$A_\cfq(\lambda)$ is a fundamental series 
		representation(?) 
	\item  $[\cfl,\cfl]\nsubseteq\cfk$ then
		$A_\cfq(\lambda)$ is non-tempered.
	\item if  $\cfu\cap\cfp=0$ 
		(for example when  $\cfq=\cfg)$,
		then  $A_\cfq(\lambda)=\lambda$.
		(at least when $\lambda$ is trivial).
\end{itemize}

\begin{defn}
	 Let $(\pi,\mathcal{H})$
	 be an irreducible unitary 
	 representation of $G$.
	 Then the smooth vectors
	 $\mathcal{H}^\infty\subset \mathcal{H}$
	  is a dense subset that is 
	  invariant under $G$,
	  and the  $K$-finite subspace
	  $\mathcal{H}^K\subset\mathcal{H}^\infty$
	  is an irreducible  $(\cfg, K)$-module
	  called the Harish-Chandra module
	  attached to  $\pi$.
\end{defn}



Let $\Lambda^*\cfp$ be the quotient of   $\Lambda^*\cfg$
by the ideal generated by  $\cfk$.
Then for an $\cfg$-module  $X$,
$\Hom(\Lambda^*\cfp,X)\subset \Hom(\Lambda^*\cfg,X)$
and the differential preserves $\Hom_\cfk(\Lambda^*\cfp,X)$
and defines  $H^*(\cfg,\cfk,X)$.

\begin{thm}
	Let $F$ be a finite-dimensional $G$-representation
	and  $\mathcal{H}$ an irreducible unitary representation.
	\begin{enumerate}[label=(\alph*)]
		\item $H^*(G,\mathcal{H}\otimes F)\cong 
			H^*(\cfg,\cfk,\mathcal{H}^K\otimes F)$.
		\item if the actions of the Casimir operator
			on $\mathcal{H}^K$ and $F$
			don't agree, then
			$H^*(\cfg,\cfk,\mathcal{H}^K\otimes F)=0$
		\item if they agree, then
			 $H^*(\cfg,\cfk,\mathcal{H}^K\otimes F)=
		  Hom_\cfk(\Lambda^*\cfp,\mathcal{H}^K\otimes F)$.
	\end{enumerate}
\end{thm}

\begin{thm}
	Let $R=\dim(\cfu\cap\cfp)$
	and $F$ be a f nite-dimensional irreducible $\cfg$-module
	of lowest weight $-\gamma\in \cfh^*$
	with respect to  $\Delta^+(\cfg,\cfh)$,
	then if $\gamma=\lambda\vert_{\cfh}$
	\[
		H^\iota(\cfg,\cfk,A_\cfq(\lambda)\otimes F)\cong
		H^{\iota-R}(\cfl,\cfk\cap\cfl,\C)\cong
		\Hom_{\cfl\cap\cfk}
		(\Lambda^{\iota-R}(\cfl\cap\cfp), \C)
	\]
	and otherwise
	$H^\iota(\cfg,\cfk,A_\cfq(\lambda)\otimes F)=0$.
\end{thm}

\begin{thm}
	Suppose $H^*(G,\pi\otimes F)\neq 0$,
	then there exists  $\cfq=\cfl+\cfu$ such that
	 \begin{enumerate}[label=(\alph*)]
		\item $F/\cfu F$ is a one-dimensional 
			unitary representation $\lambda$ of $L$.
			Let $-\lambda\colon \cfl\to \C$
			be the differential of which.
		\item  $X\cong A_\cfq(\lambda)$
	\end{enumerate}
\end{thm}

\begin{thm}
	When $G$ is simple and  $\pi$ is a nontrivial
	unitary irreducible representation
	and  $E$ is finite dimensional.
	Then  $H^i(G,\pi\otimes E)=0$ 
	for all $i<r_G$, which is the minimum dimension
	of  $\cfu\cap \cfp$.
\end{thm}
\begin{align*}
	\SL(2,\R)\colon & r_G=1\\
	\SU(p,q),p\leq q\colon& r_G=p\\
	\SO(p,q),p\leq q,2\leq q\colon& r_G=p.
\end{align*}

\subsection{Hermitian symmetric}

$\Lambda^\iota\cfp=\oplus_{p+q=\iota}(\Lambda^p\cfp^+)
\otimes(\Lambda^q\cfp^-)$ and
\[
	\Hom_\cfk(\Lambda^\iota\cfp,X)\cong
	\bigoplus_{p+q=\iota}
	\Hom_\cfk((\Lambda^p\cfp^+) \otimes(\Lambda^q\cfp^-),X)
\]
with $d=\partial+\bar{\partial}$.
Thus 
$H^i(\cfg,\cfk,X)\cong \oplus_{p+q=i}H^{p,q}(\cfg,\cfk,X)$

\begin{prop}
	Let $R^{\pm}=\dim(\cfu\cap\cfp^\pm)$, 
	then
	\[
		H^{p+R^+,p+R^-}(\cfg,\cfk,A_\cfq(\lambda)\otimes F)
		\cong H^{p,p}(\cfl,\cfl\cap\cfk,\C)
		\cong \Hom_{\cfl\cap\cfk}
		(\Lambda^{2p}(\cfl\cap\cfp),\C).
	\]
	the cohomology $H^{p,q}=0$ 
	if $p-q\neq R^+-R^-$.
\end{prop}


\subsection{discrete series}

Assume furthermore $\rank(G)=\rank(K)$
and let $T\subset K$ be a Cartan subgroup.
Define
\[
    \Delta=\Delta(G,T),
    \Delta_c=\Delta(K,T),
    P(\Delta)=
    \{\lambda\in \mathfrak{t}_\C^*\mid \langle \lambda,\alpha^\vee\rangle\in \Z\}.
\]
Then there exists a correspondence between $\hat{G}_{DS}$ and
\[
    \Xi=\{\lambda\in P(\Delta)\mid
    \langle \lambda, \alpha^\vee\rangle\neq 0 \text{ for }
    \alpha\in \Delta,\,
    \langle \lambda, \alpha^\vee\rangle>0 \text{ for }
    \alpha\in \Delta_c^+\}
\]
Given $\lambda\in\Xi$, the corresponding discrete series representation 
$\pi_\lambda$ satisfies the following properties.
\begin{enumerate}
    \item $\pi_\lambda$ has
    the infinitesimal character 
    $\lambda\in \mathfrak{t}_\C^*/W_G$.
    \item let $\Delta_\lambda^+=\{\alpha\mid \langle\lambda,\alpha^\vee\rangle>0\}$
    and define 
    $\Lambda=\lambda+\frac{1}{2}\sum_{\Delta_\lambda^+}\alpha-\sum_{\Delta_c^+}\alpha$.
    Then $\tau_\Lambda\subset \pi_\lambda$
    is the minimal $K$-type of $\pi_\lambda$,
    which has multiplicity one and highest weight $\Lambda$.
    \item let $\tau_\mu\in \hat{K}$ be of highest weight $\mu$,
    then $\tau_\mu\subset \pi_\lambda$ implies that
    $\mu=\Lambda+\sum_{\alpha\in \Delta_\lambda^+\setminus \Delta_c^+} m_\alpha\alpha$
    for $m_\alpha\in \Z_{\geq 0}$.
\end{enumerate}
Recall that there is the Harish-Chandra isomorphism
$q\colon Z(\gl)\cong \Sym(\mathfrak{t}_\C)^{W_G}$, thus for $\pi=\pi_\lambda$
\[
    \chi_\pi(z)=\langle \nu, q(z)\rangle,\quad
    \lambda\in \mathfrak{t}_\C^*/W^G
\]
and there are $\#W_G/W_K$ discrete series with the same infinitesimal character.
We call $\lambda$ the Harish-Chandra parameter and
$\Lambda$ the Blattner parameter of $\pi_\lambda$.

	If $w\in W$, define the length by
		 \[
			 \ell(w)=\#\{\alpha\in \Delta^+\mid
			 w(\alpha)\notin\Delta^+\}
		\]


\section{cohomology}

$U=X\setminus D$ where
$D=\sum_{j=1}^rD_j$ is a reduced normal crossing divisor.
Write $\iota\colon U\hookrightarrow X$ and
\[
	 \Omega_X^a(*D)=\varprojlim_\nu \Omega_X^a(\nu\cdot D)
	 =\iota_*\Omega_U^a
\]
Define $\Omega_X^a(\log D)$ as the subsheaf of $\Omega_X^a(*D)$
consists of sections  $\alpha$ such that 
$\alpha$ and  $d\alpha$ have at most simple poles along  $D$.
\begin{itemize}
	\item $\Omega_X^a(\log D)\hookrightarrow \Omega_X^a(*D)$
	\item  $\Omega_X^a(\log D)=\wedge^a \Omega_X^1(\log D)$
	\item for  $p\in X$ with local parameters
		 $f_1,\cdots,f_n$ such that 
		 $D_j$ is defined by  $f_j=0$.
		 Suppose  $p\in D_j$ for  $j=1,\cdots,s$ and 
		 $p\notin D_j$ for  $j=s+1,\cdots,r$.
		 Let
		 \[
		 	\delta_j=
			\begin{cases}
				\frac{df_j}{f_j} & j\leq s;\\
				df_j & j>s.
			\end{cases}
		 \]
		 then $\Omega^1(\log D)$
		 is locally generated by $\delta_j$
		 over  $\mathcal{O}_X$.
\end{itemize}
\begin{proof}
	Consider $\alpha\in \Omega_X^a(*D)$ written as
	 \[
		\alpha=\alpha_1+\alpha_2\wedge\frac{df_1}{f_1}
	\]
	where $\alpha_i$ are generated by wedge products
	of  $df_2,\cdots,df_n$. 
	Then $\alpha\in \Omega_X^a(\log D)$ implies
	 \[
	f_1\cdot \alpha=f_1\cdot\alpha_1+\alpha_2\wedge df_1,
	\in \Omega_X^a
	f_1\cdot d\alpha=f_1d\alpha_1+d\alpha_2\wedge df_1
	\in \Omega_X^{a+1}
	\]
	so $\alpha_2$ and  $f_1\alpha_1$ have no poles,
	then so does $\alpha_1$ since
	 \[
		 d(f_1\alpha_1)=
		 df_1\wedge\alpha_1+f_1d\alpha-d\alpha_2
	\]
\end{proof}

\begin{align}
	&\alpha\colon \Omega_X^1(\log D)\to
	\bigoplus_{j=1}^s\mathcal{O}_{D_j},\quad
	\alpha(\sum a_j\delta_j)=\bigoplus a_j\vert_{D_j}\\
	&\beta_1\colon \Omega_X^a(\log D)\to 
	\Omega_{D_1}^{a-1}(\log(D-D_1)\vert_{D_1})\quad
	\beta_1(\varphi_1+\varphi_2\wedge\frac{df_1}{f_1})
	=\sum a_I\delta_{I-\{1\}}\vert_{D_1},\\
	&\varphi_2=\sum a_I\delta_{I-\{1\}},
	\varphi_1=\sum_{1\notin I} a_I\delta_I.\\
	&\gamma_1\colon 
	\Omega_X^a(\log(D-D_1))\to 
	\Omega_{D_1}^a(\log(D-D_1)\vert_{D_1}),\quad
	\gamma_1(\sum_{1\in I}f_1a_I\delta_I+
	\sum_{1\notin I}a_I\delta_I)=
	\sum_{1\notin I}a_I\delta_I\vert_{D_1}
\end{align}

\begin{gather*}
	0\to \Omega_X^1\to \Omega_X^11(\log D)\to 
	\bigoplus_{j=1}^r\mathcal{O}_{D_j}\to 0\\
	0\to \Omega_X^a(\log(D-D_1))\to 
	\Omega_X^a(\log D)\xrightarrow{\beta}
	\Omega_{D_1}^{a-1}(\log(D-D_1)\vert_{D_1})\to 0\\
	0\to \Omega_X^a(\log D)(-D_1)\to
	\Omega_X^a(\log(D-D_1))\xrightarrow{\gamma}
	\Omega_{D_1}^a(\log(D-D_1)\vert_{D_1})\to 0
\end{gather*}

\begin{defn}
	From 
	\[
		\nabla\colon E\to \Omega_X^1(\log D)\otimes E,\quad
		\nabla(f\cdot e)=f\cdot \nabla(e)+df\otimes e
	\]
	define 
	\[
		\nabla_a\colon \Omega_X^a(\log D)\otimes E
		\to \Omega_X^{a+1}(\log D)\otimes E,\quad
		\nabla_a(\omega\otimes e)=d\omega\otimes e+
		(-1)^a\cdot \omega\wedge\nabla(e)
	\]
\end{defn}

\section{Relative de Rham complex}
$f\colon U\to V$ proper smooth morphism between
smooth  $\C$-schemes
and  $(M,\nabla)$ an algebraic differential equation on  $U$,
i.e. $M$ is locally free sheaf on  $U$ and integrable connection
 \[
	 \nabla\colon M\to \Omega_U^1\otimes_{\mathcal{O}_U}M
\]
compose with $\Omega_U^1\to \Omega_{U/V}^1$,
get relative de Rham complex, whose hypercohomology.
\[
	H^q_{dR}(U/V,(M,\nabla))=
	\mathbb{R}^qf_*(\Omega_{U/V}^*\otimes_{\mathcal{O}_U}M)
\]

Filter the absolute de Rham complex by
\[
	F^i(\Omega_U^*\otimes_{\mathcal{O}_U}M)=
	\text{image of }
	f^*(\Omega_V^i)\otimes_{\mathcal{O}_U}
	\Omega_U^{*-i}\otimes_{\mathcal{O}_U}M
	\to \Omega_U^*\otimes_{\mathcal{O}_U}M\,
	gr^i=F^i/F^{i+1}=
	f^*(\Omega_V^i)\otimes_{\mathcal{O}_U}
	\Omega_U^{*-i}\otimes_{\mathcal{O}_U}M
\]
The Gauss-Manin connection is the coboundary map of the 
long exact sequence of $\mathbb{R}^qf_*$
from 
 $0\to gr^1\to F^0/F^2\to gr^0\to 0$, since
  \begin{gather*}
	  \mathbb{R}^qf_*(gr^0)=H^q_{dR}(U/V,(M,\nabla))\\
	  \mathbb{R}^{q+1}f_*(gr^1)=
	  \Omega_V^1\otimes_{\mathcal{O}_V}H^q_{dR}(U/V,(M,\nabla))
 \end{gather*}
 (why integrable?)
 and thus $H^q_{dR}(U/V,(M,\nabla))$
 are locally free as well.

 When $(M,\nabla)=(\mathcal{O}_U,d)$,
 then  $H_{dR}(U/V)=\mathbb{R}f_*(\Omega_{U/V}^*)$
 is the Gauss-Manin connection
 and the algebraic differntial equations is called
 the Picard-Fuchs equation.

\[
	\begin{tikzcd}
		U\arrow[r,hookrightarrow]
		 \arrow[d,"f"]
		 & S \arrow[d,"\pi"]\\
		V\arrow[r,hookrightarrow] & T
	\end{tikzcd}
\]
where $V=T\setminus Y$ and  $D=\pi^{-1}(Y)$ is normal crossing.
Define the locally free sheaf 
of relative differrntials with logarithmic singularities along  $D$
by 
 \[
	 \Omega_{S/T}^1(\log D)=
	 \Omega_S^1(\log D)/\pi^*(\Omega_T^1(\log Y)),
	 \Omega_{S/T}^p(\log D)=
	 \wedge_{\mathcal{O}_S}^p\Omega_{S/T}^1(\log D)=
\]
which fits into
\[
	0\to\pi^*(\Omega_T^1(\log Y))\otimes_{\mathcal{O}_S}
	\Omega_{S/T}^{*-1}(\log D)\to 
	\Omega_S^*(\log D)\to 
	\Omega_{S/T}^*(\log )\to 0
\]

When $(M,\nabla)$ has regular singular points, i.e.
there is locally free  $\overline{M}$ on  $S$ which prolongs  $M$
and 
 \[
	 \bar{\nabla}\colon \overline{M}
	 \to \Omega_S^1(\log D)\otimes_{\mathcal{O}}\bar{M}
\]
Filter $\Omega_S^*(\log D)\otimes_{\mathcal{O}}\bar{M}$
by subcomplexes
 \[
	F^i=\text{image of }
	\pi^*(\Omega_T^i(\log Y)\otimes_{\mathcal{O}_S}
	\Omega_S^{*-i}(\log D)\otimes_{\mathcal{O}_S}\bar{M}
	\to \Omega_S^*(\log D)\otimes_{\mathcal{O}_S}\bar{M}
\]
then
$gr^i=F^i/F^{i+1}=\pi^*(\Omega_T^i(\log Y)\otimes_{\mathcal{O}_S}
	\Omega_S^{*-i}(\log D)\otimes_{\mathcal{O}_S}\bar{M}$
and $gr^0$ prolongs  $\Omega_{U/V}\otimes M$.
Same constructions provide prolongations.
So the relative de Rham coholomologies on $V$
has regular singular points.

\subsection{exponents}
The map
\[
	L_i\colon \overline{M}
	\to \Omega_S^1(\log D)\otimes_{\mathcal{O}_S}\overline{M}
	\xrightarrow{\text{residue}}
	\mathcal{O}_{D_i}\otimes_{\mathcal{O}_S}\overline{M}
\]
is $\mathcal{O}_{D_i}$-linear.
Since $D_i$ is proper, 
the characteristic polynomial
$P_i=\det(XI-L_i;\overline{M}\otimes\mathcal{O}_{D_i})\in \C[X]$ 
whose roots are called the exponents
of $(\overline{M},\bar{\nabla})$ around $D_i$

\section{Katz-Oda}

$S$ is smooth scheme over the field  $k$,
and $\EE$ is a quasi-coherent sheaf of  $\oo_S$-module.
Let  $\Der_k(\oo_S)$ be the sheaf of $k$-derivations of 
 $\oo_S$ into itself,
which is naturally a sheaf of  $k$-Lie algbras,
whiles as  $\oo_S$-module isomorphic to
$\underline{\Hom}_{\oo_S}(\Omega_{S/k}^1,\oo_S)$.
Let $\underline{\End}_k(\EE)$
be the sheaf of  $k$-linear endomorhism.

Given a connection on $\EE$,
there is a  $\oo_S$-linear map
 \[
	 \Der_k(\oo_S)\to \underline{\End}_k(\EE),\quad
	 D\mapsto
	 \tilde{D}\colon 
	 \EE\to \Omega_{S/k}^1\otimes_{\oo_S}\EE
	 \xrightarrow{D\otimes 1}
	 \oo_S\otimes_{\oo_S}\EE\cong \EE
\]

Conversely, since $S$ is smooth over  $k$,
any  $\oo_S$-linear map
$\Der_k(\oo_S)\to \underline{\End}_k(\EE)$
with  $ \tilde{D}(fe)=D(f)e+f\tilde{D}(e)$
comes from a unique connection.
The connection is integrable iff
the above map is a Lie-algebra homomorphism.

 \[
	 d_1^{p,q}\colon E_1^{p,q}=R^{p+q}\pi_*(gr^p)\to 
	 E_1^{p+1,q}=R^{p+1+q}\pi_*(gr^{p+1})
\]
is the connecting homomorphism of 
\[
	0\to gr^{p+1}\to F^p/F^{p+2}\to gr^p\to 0.
\]

local system $\EE$ and 
integrable connection on a coherent sheaves $\EE\otimes\oo_X$

\section{BBG}
For  $\rfg=\rfk+\rfp$,
let  $\Theta=\Ad_{h(i)}$
and $\mu=\mathbb{G}_m\to G_\C$,
then
 \[
	\cfg=\cfk+\cfp^++\cfp^-=
	\cfg^{0,0}+
	\cfg^{-1,1}+
	\cfg^{1,-1}
\]
on which $\mu(z)$ acts by  $z^{\pm}$.
Choose $\Delta=\Delta_c\sqcup \Delta_n$
such that  $\cfp^\pm=\oplus_{\alpha\in \Delta_n^\pm}\cfg_\alpha$.
And let $Q\subset G_\C$
be the parabolic subgrouop correspondes
to  $\cfq=\cfk+\cfp^-$.
 \[
	 \begin{tikzcd}
		 D=G/K\arrow[rr,hookrightarrow]
		 \arrow[dr]
		 && \check{D}=G_\C/Q\\
		 &
		 \exp(\cfp^+)\cong
		 \exp(\cfp^+)\cdot Q/Q\arrow[ur]&
	 \end{tikzcd}
\]

Let $W=W(\Delta)\supset W_c=W(\Delta_c)$
and  $W_n=\{w\mid w(\Delta^+)\supset W^+_c\}$ 
is a coset representative of $W/W_c$.
Given a finite dimensional
$Q$-representaiton  $V$,
form  $\underline{V}=G_\C\times V/Q$
then
$T_{\check{D}}\cong \underline{\cfg/\cfp^-}$
and $\Omega^1_{\check{D}}\cong \underline{\cfp^-}$.
\begin{thm}
	$G$-equivariant differntial
	$\underline{W}_1\to
	\underline{W}_2$ 
	correspondes
	to 
	\[
		\Hom_\cfg(
		\mathcal{U}(\cfg)\otimes
		\mathcal{U}(\cfq)W_2^*,
		\mathcal{U}(\cfg)\otimes
		\mathcal{U}(\cfq)W_1^*)
		\cong 
		\Hom_\cfq( W_2^*,
		\mathcal{U}(\cfg)\otimes
		\mathcal{U}(\cfq)W_1^*)
	\]
\end{thm}
\begin{thm}
	There exists resolution
	\[
		0\to
		\underline{V(\lambda)}\to
		K^0_\lambda\to\cdots\to
		K^i_\lambda=\bigoplus_{
		w\in W_n\mid \ell(w)=i}
		\underline{W}(w(\lambda+\rho)-\rho)\to \cdots
	\]
\end{thm}

\begin{thm}
	$H^q(\Gamma\backslash D,K^p_\lambda)
	\Longrightarrow H^{p+q}
	(\Gamma\backslash D, V(\lambda))$,
	the degeneracy used mixed Hodge 
	theory(?)
\end{thm}
\begin{rem}
	Use Bett-de Rham comparison,
	the Betti cohomology
	$H^*(\Gamma\backslash D, V(\lambda))$
	can be computed by 
	hypercohomology
	($G$-repn gives natural 
	integrable connection)
	and
	$(\cfg,K)$-cohomoloy.
	On the other hand,
	$H^*(\Gamma\backslash D, \underline{W})$
	can be computed by the
	$(\cfq,K)$-cohomoloy.
	using Dolbeaut
	\[
		0\to \oo^{hol}\to C^\infty
		\xrightarrow{\bar{\partial}}
		\bar{\Omega}^1=
		C^\infty\otimes(\cfp^-)^*
		\xrightarrow{\bar{\partial}}\cdots
	\]
	\begin{itemize}
		\item the $(\cfq,K)$-cohomology
			are only nonzero at
			one degree and 
			 $1$-dimensional.
		In cases when BGG degenerates,
		$(\cfg,K)$-cohomology 
		can be broken into
		$(\cfq,K)$-cohomologies.
		How is this related to 
		Vogan-Zuckerman 
		classification.
	\item How to see the mixed Hodge structure
		on the other cohomologies.
	\end{itemize}
\end{rem}

\subsection{example}

When $\cfg=sl_2\supset \cfq=\smat{*&*\\&*}$ 
and $n=\smat{0&\\*&0}$,
$\Delta^+=\{\alpha\}$ and $\alpha(h)=2$.
If  $W$ is finite-dimensional  $\cfq$-representation,
then  $\mathcal{U}(\cfg)\otimes_{\mathcal{U}(\cfq)}W=\mathcal{U}(n^-)W$
\[
	0\to
	\mathcal{U}(\cfg)
	\otimes_{ \mathcal{U}(\cfq)}
	\C_{\frac{-n-2}{2}\alpha}\to
	\mathcal{U}(\cfg)
	\otimes_{ \mathcal{U}(\cfq)}
	\C_{\frac{n}{2}\alpha}\to
	Sym^n\C^2\to0
\]
the weights of the middle one is
the following multiples of $\alpha$:
 \[
	-\frac{n}{2}-2,
	-\frac{n}{2}-1,
	(-\frac{n}{2},\cdots,
	\frac{n}{2}-1,
	\frac{n}{2})
\]
Note that 
if define 
$w*\lambda=w(\lambda+\rho)-\rho$,
then  $w*\frac{n}{2}=\frac{-n-2}{2}$,
and $\Omega$ acts by
 $\frac{1}{2}\lambda^2+\lambda$
 \[
	 \frac{1}{2}(-n-2)^2+(-n-2)=
	 \frac{1}{2}n^2+n
 \]

\subsection{example}

When $G=\SL_2$, define
 \[
	h\colon \C^\times\to \SL_2\mid
	x+iy\mapsto
	\frac{1}{\sqrt{x^2+y^2}}
	\smat{x&y\\-y&x}
\]
the Cartan decomposition is 
$\rfg=\smat{&c\\-c&}+\smat{a&b\\b&-a}$.
\begin{gather*}
	(x+iy)\text{-eigenvector}\colon
	\smat{-i\\1}
	(x-iy)\text{-eigenvector}\colon
	\smat{i\\1}
\end{gather*}
and therefore $\cfg=\cfk+\cfp^++\cfp^-$ is
\[
	\Ad_{\smat{-i&i\\1&1}}
	\left[
	\smat{ic&\\&-ic}+
	\smat{0&A\\&0}+
	\smat{0&\\B&0}+
	\right]
\]
with $\alpha\colon \smat{&-c\\c&}=
	\Ad_{\smat{-i&i\\1&1}}\smat{ic&\\&-ic}
	\mapsto 2ic $ 
	Then  $Q=\Ad_{\smat{-i&i\\1&1}}\smat{*&\\*&*}$, with
\[
	D=
	\SL_2(\R)/\SO(2)\to
	\SL_2(\C)/Q
	\xrightarrow{\cdot\smat{-i&i\\1&1}}
	\SL_2(\C)/\smat{*&\\*&*}\to
	P^1,\quad
	\smat{a&b\\c&d}\mapsto [b:d]
\]
	so $\smat{a&b\\c&d}\in D\mapsto 
	\frac{a+bi}{c+di}$

\subsection{sp4}

For $\mu=(k_1,k_2), k_1\geq k_2\geq 0$,
find  $\nu$
such that  $W_\nu\cong W_{w*\mu}^\vee$
with $w\in W_n$.
Note that  $\rho=(2,1)$,
then 
 \begin{align*}
	 w=1\,&\ell(w)=0 & &\nu=(-k_2,-k_1)\\
	 w=\pm_2\,& \ell(w)=1\,
	 w*(k_1+2,k_2+1)-(2,1)
	 =(k_1,-k_2-2) & & \nu=(k_2+2,-k_1)\\
	 w=\pm_1\times\,& \ell(w)=2\,
	 w*(k_1+2,k_2+1)-(2,1)
	 =(k_2-1,-k_1-3) & & \nu=(k_1+3,-k_2+1)\\
	 w=\pm_2\pm_1\times\, & \ell(w)=3,
	 w*(k_1+2,k_2+1)-(2,1)
	 =(-k_2-3,-k_1-3) & & \nu=(k_2+3,k_1+3).
\end{align*}

\subsection{Katz}

Same notation,
for $U\subset \check{D}$ is open,
let $ \tilde{U}\subset G_\C$ be the preimage.
If also $U\subset D$,
let $ U_1\subset G$ be the preimage.
Given a $Q$-representation,
 \begin{align*}
	 \Gamma(U,W)&=
	 \{
	 f\colon \tilde{U}\xrightarrow{holo}W\mid
	 f(gq)=q^{-1}\cdot f(g)\Longleftrightarrow
	 R(X)(f)(g)=-X(f(g)), X\in \cfq
	 \}\\
	 C^\infty(U,W)&=
	 \{
	 f\colon \tilde{U}\xrightarrow{C^\infty}W\mid
	 f(gq)=q^{-1}\cdot f(g)\Longleftrightarrow
	 R(X)(f)(g)=-X(f(g)), X\in \cfq
	 \}\\
\end{align*}

And when $U\subset D$
 \[
	 C^\infty(U,W)=
	 \{
	 f\colon U_1\xrightarrow{C^\infty}W\mid
	 f(gk)=k^{-1}\cdot f(g)
	 \}
\]
and $f(g)$ is holomophic if and only if
$R(X)(f)(g)=-X(f(g))$ for all  $X\in \cfp^-$.

The holomorphic de Rham
$\Omega^p_{\check{D}}$
correspondes to $\wedge^p\cfp^-$,
and  the  $C^\infty$ double complex
$\Omega^{p,q}_{\check{D}}$ 
correspondes to (the $K$-representation)
 $\wedge^p\cfp^-\otimes\wedge^q\cfp^+$,
if
 \[
	f\colon U_1\to 
	\wedge^p\cfp^-\otimes\wedge^q\cfp^+
\]
then
\[
	d'f=\sum_{\alpha\in\Delta_n^+}
	E_{-\alpha}\circ R(X_\alpha)(f),\quad
	d''f=\sum_{\alpha\in\Delta_n^-}
	E_{-\alpha}\circ R(X_\alpha)(f).
\]

Let $p\in \check{D}$ be the base point $Q$,
$f\in \underline{W}_p$,
then
\[
	\underline{W}_p\to
	\Hom_\cfq(\mathcal{U}(\cfg),W),\quad
	f\mapsto 
	[Z\mapsto R(Z)(f)(1)]
\]
is $\cfg$-linear.
 \begin{itemize}
	 \item $R(-ZX)(f)(1)=X(R(Z)(f)(1))$
		 for  $X\in\cfq$.
	 \item $R(Z)(L(X)f)(1)=R(-X.Z)(f)(1)$.
\end{itemize}
Then a homogeneous holomorphic differential 
operator between
$\underline{W}_1\to \underline{W}_2$
correspondes to 
 \begin{multline*}
 	\cfg\text{-linear }
	\Hom_\cfq(\mathcal{U}(\cfg),W_1)\to
	\Hom_\cfq(\mathcal{U}(\cfg),W_2)\\
	\cong 
	\cfq\text{-linear }
	\Hom_\cfq(\mathcal{U}(\cfg),W_1)\to
	W_2\\
	\cong 
	\Hom_\cfq(
	W_2^*,
	\mathcal{U}(\cfg)
	\otimes_{\mathcal{U}(\cfq)}W_1^*)
 \end{multline*}

Let $w_0$ be the longest root 
with  $w_0(\Delta^+)=\Delta^-$
and  $W(\mu)^*$ 
has lowest weight  $-\mu$.

For  $V=V(\lambda)^*=V(-w_0(\lambda))$
with lowest weight  $-\lambda$,
the BGG resolution starts with $\lambda=0$ and
bar resolution
\[
0\leftarrow \C	
\leftarrow L_0
\leftarrow L_1
\leftarrow \cdots
\leftarrow L_i
=\mathcal{U}(\cfg)\otimes_{\mathcal{U}(\cfq)}
\wedge^i(\cfg/\cfq)
\leftarrow \cdots
\]

\[
	V(\lambda)\leftarrow
	K^i_\lambda=
	\bigoplus W(w(\lambda+\rho)-\rho)
\].
The dual BGG complex
\[
	\underline{V}(\lambda)
	\to K^i_\lambda
	\bigoplus \underline{W}
	(w(\lambda+\rho)-\rho)
\]
is a subcomplex 
of the de Rham resolution
of the locally constant sheaf of $V(\lambda)$

 \begin{rem}
	When using the Dolbeault complex
	to compute $H^q(X,K^p_\lambda)$,
	use 
	 \[
		K^*_\lambda\otimes \bar{K}_0^*
		\subset
		K^*_\lambda\otimes \bar{\Omega}^*
		\subset 
		\Omega^{*,*}(V(\lambda))
	\]
	When $\Gamma\subset G$
	is cocompact arithmetic
	without torsion,
	 \[
		 H^n(\Gamma,V(\lambda))=
		 \bigoplus_{\ell(w)+\ell(w')=n}
		 H^{w,w'}_\lambda,\quad
		 H^q(X, \underline{W}
		 (w(\lambda+\rho)-\rho))=
		 \bigoplus_{\ell(w')=q}
		 H^{w,w'}_\lambda,\quad
	\]
	where 
	$H^{w,w'}_\lambda$ 
	consists of  $C^\infty$
	 \[
		f\colon \Gamma\backslash G\to
		W(w(\lambda+\rho)-\rho)\otimes
		\overline{W(w\rho-\rho)}
	\]

	When not cocompact,
	$H^*(\Gamma\backslash V,
	\underline{V}(\lambda))$
	can be computed,
	using similar decomposition
	for the canonical extensions
	of  $\underline{W}$; 
	$H^*_c(\Gamma\backslash V,
	\underline{V}(\lambda))$
	can be computed,
	using similar decomposition
	for the subcanonical extensions.
\end{rem}

\section{mixed hodge}

Define filtrations on 
$H^*(V,\C)=\mathbb{H}^*
(\bar{V},\Omega^*_{\bar{V}}(\log D))$
\[
	F^p\Omega^i_{\bar{V}}(\log D)=
	\begin{cases}
		\Omega^i_{\bar{V}}(\log D)&
		i\geq p\\
		0 & i<p.
	\end{cases}
	\text{ filtration bete}
\]
then define
$F^pH^*(V,\C)=\text{image}
(\alpha_p\colon\mathbb{H}^*
(\bar{V},F^p\Omega^*_{\bar{V}}(\log D))\to
\mathbb{H}^*
(\bar{V},\Omega^*_{\bar{V}}(\log D))
)$.

On the other hand, define the weight filtration
\[
	W_m\Omega^*_{\bar{V}}(\log D)=
	\begin{cases}
		0 & m<0\\
		\Omega^p_{\bar{V}}(\log D)&
		p<m\\
		\Omega^{p-m}_{\bar{V}}\wedge
		\Omega^m_{\bar{V}}(\log D)&
		0\leq m\leq p
	\end{cases}
\]
then define
$W_m\mathbb{H}^k
(\bar{V},\Omega^*_{\bar{V}}(\log D))=
\text{image}(
\beta_m\colon
\mathbb{H}^k
(\bar{V},W_{m-k}\Omega^*_{\bar{V}}(\log D))\to
\mathbb{H}^k
(\bar{V},\Omega^*_{\bar{V}}(\log D))
)$ 

Let $D^{[m]}$ be the union of all $m$-fold
intersetion of  $D_i$
 \[
	 W_m\Omega_{\bar{V}}^*(\log D)
	 \to o
	 (a_n)_*\Omega_{D^{[m]}}^*,\quad
	 Gr^W_m \Omega_{\bar{V}}^*(\log D) \to 
	 (a_n)_*\Omega_{D^{[m]}}^*[-m]
\]
Then
$W_kH^k(V,Q)=image(
H^k(\bar{V},Q)\to H^k(V,Q))$
and 
\[
	Res[i]\colon 
	W_{k+i} H^k(V,\C)\to 
	H^{k-i}(D^{[i]},\C),\quad i\geq0
\]
with the kernel being $W_{k+i+1}$




%\bib ingraphystyle{amsalpha}
%\bib raphy{biblio}
\end{document}

